%\usepackage[fleqn]{amsmath}
\usepackage{amssymb,graphicx,color,graphicx,slashed, microtype, parskip, enumitem, extarrows, needspace}
\usepackage[top=1.7cm, bottom=1.5cm, right=2.0cm, left=2.0cm, heightrounded, marginparwidth=5cm, marginparsep=0.5cm]{geometry}
\usepackage{amsmath}
\usepackage{amssymb}
% \usepackage{mathpazo}
\usepackage{amsfonts}
\usepackage{amsmath,amscd}
\usepackage{geometry}
\usepackage{mathrsfs}
\usepackage{lmodern}
% \usepackage{fourier}
% \usepackage{ccfonts}
\usepackage[T1]{fontenc}
% \usepackage{antpolt}
% \usepackage{utopia}
% \usepackage{PTSerif}
\usepackage{cite}
\usepackage{changepage}
\usepackage{marvosym}
\usepackage{manfnt}
\usepackage{framed}
\usepackage{multirow}
\usepackage{etoc}
% \usepackage{ctex} % Chinese 

% \geometry{
	% paper=a4paper,
	% top=3cm,
	% inner=2.54cm,
	% outer=2.54cm,
	% bottom=3cm,
	% headheight=5ex,
	% headsep=5ex,
% }
\usepackage[margin=1.5in]{geometry}
\hbadness = 10000
\hfuzz=100pt 
\usepackage{marginnote}
\renewcommand*{\marginfont}{\footnotesize}
\usepackage{hyperref}
\hypersetup{colorlinks=true, urlcolor=NavyBlue, bookmarksdepth=3, citecolor=ForestGreen, linkcolor=black}
\makeatletter\newcommand{\@minipagerestore}{\setlength{\parskip}{\medskipamount}}\makeatother
\usepackage{titletoc}
\titlecontents{section}[3em]{}{\bf\contentslabel{2.33em}}{}{\,\titlerule*[0.5pc]{$\cdot$}\contentspage}[\vspace{1.1ex}]
\newcommand{\intro}[1]{\rightline{\parbox[t]{5cm}{\footnotesize \fangsong\quad\quad #1 }}}
% \renewcommand{\thesection}{\S\arabic{section}}
% =============== Index ===========================
\usepackage[nonewpage]{imakeidx}
\makeindex
% =============== Color Definitions ===============
\usepackage[svgnames]{xcolor}
\colorlet{ColorTitle}{Black}
\colorlet{ColorSectionName}{Black}
\colorlet{ColorBoxFG}{Gray}
\colorlet{ColorBoxText}{Black}
\colorlet{ColorBoxBG}{White}
\definecolor{browna}{RGB}{16, 17, 131}
% \definecolor{brownb}{RGB}{16,0.69,0.65}
\definecolor{bluea}{RGB}{53, 90, 142}
\definecolor{mygray}{RGB}{175, 176, 178}
\definecolor{mygreen}{RGB}{230, 230, 230}
\definecolor{myblue}{RGB}{190, 192, 191}
\definecolor{myGreen}{RGB}{68, 128, 130}
\definecolor{myexamplebg}{RGB}{244,251,248}
\definecolor{myexamplefr}{RGB}{166, 219, 214}
\definecolor{myexampleti}{RGB}{69, 131, 134}
% =============== Title Style ===============
% \usepackage{titling} % Allows custom title configuration
% \newcommand{\HorRule}{\color{ColorTitle}\rule{\linewidth}{1pt}} % Defines the gold horizontal rule around the title
% \pretitle{
%     \vspace{-50pt} % Move the entire title section up
%     \HorRule\vspace{9pt} % Horizontal rule before the title
%     \fontsize{27}{36}\usefont{OT1}{phv}{b}{n}\selectfont
%     \color{ColorTitle} % Text colour for the title and author(s)
% }
% \posttitle{\par\vskip 15pt} % Whitespace under the title
% \preauthor{\fontsize{17}{0}\usefont{OT1}{phv}{m}{n}\selectfont\color{ColorTitle}} % Anything that will appear before \author is printed
% \postauthor{\par\HorRule}
% \newcommand{\COURSENAME}{\textcolor{black}{Abstract Algebra 2023}}
% % \newcommand{\COURSENAME}{\href{http://phyw.people.ust.hk/teaching/PHYS2022-2015/}{\textcolor{black}{Algebra 2023}}}
% \newcommand{\HTG}{\textcolor{black}{H.-T.~Guo}}
% % \newcommand{\PHYS}{\href{https://lxy.njupt.edu.cn/}{\textcolor{black}{School of Electronic Engineering}}}
% \newcommand{\PHYS}{\href{https://lxy.njupt.edu.cn/}{\textcolor{black}{School of Science}}}
% \newcommand{\NJUPT}{\href{https://www.njupt.edu.cn/}{\textcolor{black}{NJUPT}}}
% \author{\COURSENAME, \HTG, \PHYS, \NJUPT}
% \date{}
% =============== Section Name Style ===============
\usepackage{titlesec}
\titleformat{\section}
    {\fontsize{15}{20}\usefont{OT1}{phv}{b}{n}\color{ColorSectionName}}
    {\thesection}{1em}{}
    [{\vspace{0.2cm}\titlerule[0.8pt]}]
\titleformat{\subsection}
    {\fontsize{14}{20}\usefont{OT1}{phv}{m}{n}\color{ColorSectionName}}
    {\thesubsection}{1em}{}
\titleformat{\subsubsection}
    {\fontsize{12}{20}\usefont{OT1}{phv}{m}{n}\color{ColorSectionName}}
    {}{0em}{}
\setcounter{secnumdepth}{4}

%customise the section titles
% \titleformat{\section}[block]
%   {\normalfont\fontsize{20.74}{16}\scshape\filcenter}{\thesection}{1em}{}
% \titleformat{\subsection}
%   {\normalfont\fontsize{14}{12}\scshape}{\thesubsection}{1em}{}
% \titleformat{\subsubsection}
%   {\normalfont\fontsize{12}{9}\scshape}{\thesubsubsection}{1em}{}

% =============== Box Style ===============
\usepackage[most]{tcolorbox}
\newtcolorbox{tbox}[1]{
    colback=ColorBoxBG, colframe=ColorBoxFG, coltext=ColorBoxText,
    sharp corners, enhanced, breakable, parbox=false,
    before skip=1em, after skip=1em,
    title={#1}, fonttitle=\usefont{OT1}{phv}{b}{n}, 
    attach boxed title to top left={yshift=-0.1mm}, boxed title style={sharp corners, colback=ColorBoxFG, left=0.405cm},
    rightrule=-1pt,toprule=-1pt, bottomrule=-1pt
}
\newtcolorbox{mtbox}[1]{
    colback=ColorBoxBG, colframe=ColorBoxFG, coltext=ColorBoxText,
    sharp corners, enhanced, breakable, parbox=false,
    before skip=1em, after skip=1em,
    title={#1}, fonttitle=\usefont{OT1}{phv}{b}{n},
    attach boxed title to top left={yshift=-0.1mm}, boxed title style={sharp corners, colback=ColorBoxFG, left=0.15cm},
    rightrule=-1pt,toprule=-1pt, bottomrule=-1pt, 
    left=0.5em
}

% =============== tikz has to be loaded after xcolor
\usepackage{tikz}
\usepackage{tikz-cd}
\newcommand*\enumlabel[1]{\tikz[baseline=(char.base)]{
			\node[shape=rectangle,inner sep=2pt,fill=ColorBoxFG] (char) 
			{\fontsize{7}{20}\usefont{OT1}{phv}{b}{n}{\textcolor{ColorBoxBG}{#1}}};}}
% =============== Useful shortcuts ===============
\newcommand\wref[1]{{\hypersetup{linkcolor=white}\ref{#1}}}  
\newcommand{\textbox}[2]{
    \begin{tbox}{#1}
        #2
    \end{tbox}
}
\newcommand{\mtextbox}[2]{\marginnote{
    \begin{mtbox}{#1}
        #2
    \end{mtbox}}
}
\newcommand{\mnewline}{\vspace{0.5em}\newline}
\newcommand{\titem}[1]{
    \begin{itemize}[label=\color{ColorBoxFG}$\blacktriangleright$, leftmargin=0mm, labelsep=0.27cm, topsep=0.5em
        %, itemsep=1ex
        ]
        #1
    \end{itemize}
}
\newcommand{\mtitem}[1]{
    \begin{itemize}[label={\color{ColorBoxFG}$\blacktriangleright$}, leftmargin=0mm, labelsep=1mm, topsep=0.5em
        %, itemsep=1ex
        ]
        #1
    \end{itemize}
}
\newcommand{\itembox}[3]{
    \begin{tbox}{#1}
        #2
        \titem{#3}
    \end{tbox}
}
\newcommand{\mitembox}[3]{
    \marginnote{
    \begin{mtbox}{#1}
        #2
        \mtitem{#3}
	\end{mtbox}
    }
}
\newcommand{\tenum}[1]{
    \begin{enumerate}[label=\protect\enumlabel{\arabic*}, leftmargin=0mm, labelsep=0.265cm, topsep=0.5em
        %, itemsep=1ex
        ]
        #1
    \end{enumerate}
}

\newcommand{\enumbox}[3]{
    \begin{tbox}{#1}
        #2
        \tenum{#3}
    \end{tbox}
}

\newcommand{\twocol}[5]{
    \begin{minipage}[t][][b]
        {#1\textwidth}
        #4        
    \end{minipage}
    \hspace{#2\textwidth}
    \begin{minipage}[t][][b]
        {#3\textwidth}
        #5
    \end{minipage}
}

\newcommand{\cg}[2]{
    \begin{center}
        \includegraphics[width=#1\textwidth]{#2}
    \end{center}
}
\newcommand{\tbar}{
    ~\newline
    {\color{ColorBoxFG}
    \hbox to 0.15\textwidth{\leaders\hbox to 5pt{\hss  \hss}\hfil} 
    \hbox to 0.7\textwidth{\leaders\hbox to 5pt{\hss . \hss}\hfil}}
    \mnewline
}
%----------------------------------------------------------------------------------------
%	Basic Math Notation Definition
%----------------------------------------------------------------------------------------
% Some commands that I am used to.
% Well-known algebraic structures
\newcommand{\N}{\ensuremath{\mathbb{N}}}
\newcommand{\Z}{\ensuremath{\mathbb{Z}}}
\newcommand{\Q}{\ensuremath{\mathbb{Q}}}
\newcommand{\R}{\ensuremath{\mathbb{R}}}
\newcommand{\CC}{\ensuremath{\mathbb{C}}}
\newcommand{\F}{\ensuremath{\mathbb{F}}}
\newcommand{\A}{\ensuremath{\mathbb{A}}}
\renewcommand{\P}{\mathbb{P}}
\newcommand{\E}{\mathbb{E}}

% Algebra

\newcommand{\gr}{\operatorname{gr}}
\newcommand{\Ass}{\operatorname{Ass}}
\newcommand{\topwedge}{\ensuremath{\bigwedge^{\mathrm{max}}}}
\newcommand{\rank}{\operatorname{rk}}
\newcommand{\Aut}{\operatorname{Aut}}
\newcommand{\Isom}{\operatorname{Isom}}
\newcommand{\Hm}{\operatorname{H}}  % Homology/cohomology
\newcommand{\Tr}{\operatorname{Tr}}	% trace
\newcommand{\Nm}{\operatorname{N}}	% norm
\newcommand{\Ann}{\operatorname{Ann}}
\newcommand{\Resprod}{\ensuremath{{\prod}'}}
\newcommand{\Sym}{\operatorname{Sym}}
\newcommand{\ord}{\operatorname*{ord}}
\newcommand{\trdeg}{\operatorname{tr.deg}}
\newcommand{\Gras}{\ensuremath{\mathbf{G}}}	% Grassmannians
\newcommand{\WittV}{\operatorname{W}}	% Witt vectors

% Analysis
\newcommand{\dd}{\mathop{}\!\mathrm{d}}
\newcommand{\champ}[1]{\ensuremath{\frac{\partial}{\partial #1}}}
\newcommand{\norme}[1]{\ensuremath{\| #1 \|}}
\newcommand{\normeL}[2]{\ensuremath{\| #2 \|_{L^{#1}}}}
\newcommand{\normeLs}[3]{\ensuremath{\| #3 \|_{L^{#1}, #2}}}

% General things...
\newcommand{\ceil}[1]{\ensuremath{\lceil #1 \rceil}}
\newcommand{\lrangle}[1]{\ensuremath{\left\langle #1 \right\rangle}}
\newcommand{\mes}{\operatorname{vol}}
\newcommand{\sgn}{\operatorname{sgn}}
\newcommand{\Stab}{\operatorname{Stab}}
\newcommand{\pr}{\ensuremath{\mathbf{pr}}} % projection morphism

% Categorical Terms (in my view)
\newcommand{\Obj}{\operatorname{Ob}}	% Objects
\newcommand{\Mor}{\operatorname{Mor}}	% Morphisms
\newcommand{\cate}[1]{\ensuremath{\mathsf{#1}}}	% Font series for categories
\newcommand{\dcate}[1]{\ensuremath{\text{-}\mathsf{#1}}}	% Categories with a pre-dash
\newcommand{\cated}[1]{\ensuremath{\mathsf{#1}\text{-}}}	% Categories with a post-dash
\newcommand{\identity}{\ensuremath{\mathrm{id}}}
\newcommand{\prolim}{\ensuremath{\underleftarrow{\lim}}}
\newcommand{\indlim}{\ensuremath{\underrightarrow{\lim}}}
\newcommand{\Hom}{\operatorname{Hom}}
\newcommand{\iHom}{\ensuremath{\EuScript{H}\mathrm{om}}}
\newcommand{\End}{\operatorname{End}}
\newcommand{\rightiso}{\ensuremath{\stackrel{\sim}{\rightarrow}}}
\newcommand{\longrightiso}{\ensuremath{\stackrel{\sim}{\longrightarrow}}}
\newcommand{\leftiso}{\ensuremath{\stackrel{\sim}{\leftarrow}}}
\newcommand{\longleftiso}{\ensuremath{\stackrel{\sim}{\longleftarrow}}}
\newcommand{\utimes}[1]{\ensuremath{\overset{#1}{\times}}}
\newcommand{\dtimes}[1]{\ensuremath{\underset{#1}{\times}}}
\newcommand{\dotimes}[1]{\ensuremath{\underset{#1}{\otimes}}}
\newcommand{\dsqcup}[1]{\ensuremath{\underset{#1}{\sqcup}}}
\newcommand{\munit}{\ensuremath{\mathbf{1}}} % unit in a monoidal category
\newcommand{\Yinjlim}{\ensuremath{\text{\textquotedblleft}\varinjlim\text{\textquotedblright}}} % injective limit in the Yoneda category
\newcommand{\Yprojlim}{\ensuremath{\text{\textquotedblleft}\varprojlim\text{\textquotedblright}}} % projective limit in the Yoneda category

% Homological Algebra
\newcommand{\Ker}{\operatorname{ker}}
\newcommand{\Coker}{\operatorname{coker}}
\newcommand{\Image}{\operatorname{im}}
\newcommand{\Coim}{\operatorname{coim}}
\newcommand{\Ext}{\operatorname{Ext}}
\newcommand{\Tor}{\operatorname{Tor}}
\newcommand{\otimesL}{\ensuremath{\overset{\mathrm{L}}{\otimes}}}

% Geometry
\newcommand{\Der}{\operatorname{Der}}
\newcommand{\Lie}{\operatorname{Lie}}
\newcommand{\Ad}{\operatorname{Ad}}
\newcommand{\ad}{\operatorname{ad}}
\newcommand{\Frob}{\operatorname{Fr}}
\newcommand{\Spec}{\operatorname{Spec}}
\newcommand{\MaxSpec}{\operatorname{MaxSpec}}
\newcommand{\PP}{\ensuremath{\mathbb{P}}}
\newcommand{\mult}{\operatorname{mult}}
\newcommand{\divisor}{\operatorname{div}}
\newcommand{\Gm}{\ensuremath{\mathbb{G}_\mathrm{m}}}
\newcommand{\Ga}{\ensuremath{\mathbb{G}_\mathrm{a}}}
\newcommand{\Pic}{\operatorname{Pic}}
\newcommand{\Supp}{\operatorname{Supp}}
\newcommand{\Res}{\operatorname{Res}}

% Groups
\newcommand{\Gal}{\operatorname{Gal}}
\newcommand{\GL}{\operatorname{GL}}
\newcommand{\SO}{\operatorname{SO}}
\newcommand{\Or}{\operatorname{O}}
\newcommand{\GSpin}{\operatorname{GSpin}}
\newcommand{\Spin}{\operatorname{Spin}}
\newcommand{\UU}{\operatorname{U}}
\newcommand{\SU}{\operatorname{SU}}
\newcommand{\PGL}{\operatorname{PGL}}
\newcommand{\PSL}{\operatorname{PSL}}
\newcommand{\SL}{\operatorname{SL}}
\newcommand{\Sp}{\operatorname{Sp}}
\newcommand{\GSp}{\operatorname{GSp}}
\newcommand{\PSp}{\operatorname{PSp}}
\newcommand{\gl}{\ensuremath{\mathfrak{gl}}}
\newcommand{\sli}{\ensuremath{\mathfrak{sl}}}
\newcommand{\so}{\ensuremath{\mathfrak{so}}}
\newcommand{\spin}{\ensuremath{\mathfrak{spin}}}
\newcommand{\syp}{\ensuremath{\mathfrak{sp}}}
\newcommand{\Ind}{\operatorname{Ind}}

%----------------------------------------------------------------------------------------
%	Arrow Definition
%----------------------------------------------------------------------------------------
% 以下用 tikz 定义可伸缩箭头, 不用 amsmath 和 extarrows 的版本以免 unicode-math 产生问题. 代码借自 Antal Spector-Zabusky
% 重定义 \xrightarrow[below]{above}
\makeatletter
\newbox\xratbelow
\newbox\xratabove
\renewcommand{\xrightarrow}[2][]{%
	\setbox\xratbelow=\hbox{\ensuremath{\scriptstyle #1}}%
	\setbox\xratabove=\hbox{\ensuremath{\scriptstyle #2}}%
	\pgfmathsetlengthmacro{\xratlen}{max(\wd\xratbelow, \wd\xratabove) + .6em}%
	\mathrel{\tikz [->, baseline=-.75ex]
		\draw (0,0) -- node[below=-2pt] {\box\xratbelow}
		node[above] {\box\xratabove}
		(\xratlen,0) ;}}
% 重定义 \xlefttarrow[below]{above}
\renewcommand{\xleftarrow}[2][]{%
	\setbox\xratbelow=\hbox{\ensuremath{\scriptstyle #1}}%
	\setbox\xratabove=\hbox{\ensuremath{\scriptstyle #2}}%
	\pgfmathsetlengthmacro{\xratlen}{max(\wd\xratbelow, \wd\xratabove) + .6em}%
	\mathrel{\tikz [<-, baseline=-.75ex]
		\draw (0,0) -- node[below] {\box\xratbelow}
		node[above] {\box\xratabove}
		(\xratlen,0) ;}}
% 重定义 \xleftrightarrow[below]{above}
\renewcommand{\xleftrightarrow}[2][]{%
	\setbox\xratbelow=\hbox{\ensuremath{\scriptstyle #1}}%
	\setbox\xratabove=\hbox{\ensuremath{\scriptstyle #2}}%
	\pgfmathsetlengthmacro{\xratlen}{max(\wd\xratbelow, \wd\xratabove) + .6em}%
	\mathrel{\tikz [<->, baseline=-.75ex]
		\draw (0,0) -- node[below] {\box\xratbelow}
		node[above] {\box\xratabove}
		(\xratlen,0) ;}}
% 重定义 \xhookrightarrow[below]{above}, 使用 tikz-cd 的 hookrightarrow
\renewcommand{\xhookrightarrow}[2][]{%
	\setbox\xratbelow=\hbox{\ensuremath{\scriptstyle #1}}%
	\setbox\xratabove=\hbox{\ensuremath{\scriptstyle #2}}%
	\pgfmathsetlengthmacro{\xratlen}{max(\wd\xratbelow, \wd\xratabove) + .6em}%
	\mathrel{\tikz [baseline=-.75ex]
		\draw (0,0) edge[commutative diagrams/hookrightarrow] node[below] {\box\xratbelow}
		node[above] {\box\xratabove}
		(\xratlen,0) ;}}
% 重定义 \xhooklefttarrow[below]{above}, 使用 tikz-cd 的 hookleftarrow
\renewcommand{\xhookleftarrow}[2][]{%
	\setbox\xratbelow=\hbox{\ensuremath{\scriptstyle #1}}%
	\setbox\xratabove=\hbox{\ensuremath{\scriptstyle #2}}%
	\pgfmathsetlengthmacro{\xratlen}{max(\wd\xratbelow, \wd\xratabove) + .6em}%
	\mathrel{\tikz [baseline=-.75ex]
		\draw (0,0) edge[commutative diagrams/hookleftarrow] node[below] {\box\xratbelow}
		node[above] {\box\xratabove}
		(\xratlen,0) ;}}

% 重定义 \xmapsto[below]{above}, 使用 tikz-cd 的 mapsto
\renewcommand{\xmapsto}[2][]{%
	\setbox\xratbelow=\hbox{\ensuremath{\scriptstyle #1}}%
	\setbox\xratabove=\hbox{\ensuremath{\scriptstyle #2}}%
	\pgfmathsetlengthmacro{\xratlen}{max(\wd\xratbelow, \wd\xratabove) + .6em}%
	\mathrel{\tikz [baseline=-.75ex]
		\draw (0,0) edge[commutative diagrams/mapsto] node[below] {\box\xratbelow}
		node[above] {\box\xratabove}
		(\xratlen,0) ;}}

% 定义 \xlongequal[below]{above}, 使用 tikz-cd 的等号
\newcommand{\xlongequal}[2][]{%
	\setbox\xratbelow=\hbox{\ensuremath{\scriptstyle #1}}%
	\setbox\xratabove=\hbox{\ensuremath{\scriptstyle #2}}%
	\pgfmathsetlengthmacro{\xratlen}{max(\wd\xratbelow, \wd\xratabove) + .6em}%
	\mathrel{\tikz [baseline=-.75ex]
		\draw (0,0) edge[commutative diagrams/equal] node[below] {\box\xratbelow}
		node[above] {\box\xratabove}
		(\xratlen,0) ;}}
\makeatother

%----------------------------------------------------------------------------------------
%	Definition
%----------------------------------------------------------------------------------------
\usepackage[theorems]{tcolorbox}
\newtcbtheorem[number within=section]%
{Definition} % \begin..
{Definition} % Title
{} % Style - default
{def} % label prefix; cite as ``theo:yourlabel''

%----------------------------------------------------------------------------------------
%	Definition-Theorem
%----------------------------------------------------------------------------------------
\usepackage[theorems]{tcolorbox}
\newtcbtheorem[number within=section]%
{defthm} % \begin..
{Definition-Theorem} % Title
{} % Style - default
{def} % label prefix; cite as ``theo:yourlabel''

%----------------------------------------------------------------------------------------
%	Definition-Proposition
%----------------------------------------------------------------------------------------
\usepackage[theorems]{tcolorbox}
\newtcbtheorem[number within=section]%
{defprop} % \begin..
{Definition-Proposition} % Title
{} % Style - default
{defprop} % label prefix; cite as ``theo:yourlabel''

%----------------------------------------------------------------------------------------
%	Theorem version 1
%----------------------------------------------------------------------------------------
\usepackage[framemethod=TikZ]{mdframed}
\newcounter{Thm}[section]
\renewcommand{\theThm}{\arabic{section}.\arabic{Thm}}
\newenvironment{Thm}[1][]{
	\refstepcounter{Thm}
	\mdfsetup{
		frametitle={
			\tikz[baseline=(current bounding box.east), outer sep=0pt]
			\node[anchor=east,rectangle,fill=myblue]
			{\strut Theorem~\theThm\ifstrempty{#1}{}{:~#1}};},
		innertopmargin=10pt,linecolor=myblue,
		linewidth=2pt,topline=true,
		frametitleaboveskip=\dimexpr-\ht\strutbox\relax
	}
	\begin{mdframed}[]\relax
}{\end{mdframed}}

%----------------------------------------------------------------------------------------
%	Theorem version 2
%----------------------------------------------------------------------------------------
\newtcbtheorem[number within=section]{theorem}{Theorem}{
  enhanced,
  sharp corners,
  attach boxed title to top left={
    yshifttext=-1mm
  },
  colback=white,
  colframe=browna,
  % fonttitle=\bfseries,
  fonttitle = \bfseries\sffamily,
  boxed title style={
    sharp corners,
    size=small,
    colback=browna,
    colframe=browna,
  } 
}{thm}

%----------------------------------------------------------------------------------------
%	Remark
%----------------------------------------------------------------------------------------
%\usepackage{amsthm}
%\theoremstyle{remark}
%\newtheorem{remark}[]{\bfseries Remark}          % <- This will work

\newtcbtheorem[number within=section]{Remark}{Remark}{enhanced, breakable,
    colback = white, coltitle=black, colframe = black!10!white, colbacktitle = black!5!white,
    attach boxed title to top left = {yshift = -2mm, xshift = 5mm},
   boxed title style = {sharp corners},
   fonttitle = \bfseries, separator sign = {.}, label type=remark}{re}
%----------------------------------------------------------------------------------------
%	Lemma
%----------------------------------------------------------------------------------------
% \newtcbtheorem[no counter]{Lemma}{Lemma}{
%   enhanced,
%   sharp corners,
%   attach boxed title to top left={
%     yshifttext=-1mm
%   },
%   colback=white,
%   colframe=myGreen,
%   fonttitle=\bfseries,
%   coltitle=white,
%   boxed title style={
%     sharp corners,
%     size=small,
%     colback=myGreen,
%     colframe=myGreen,
%   } 
% }{lem}
\newcounter{Lem}[section]
\renewcommand{\theLem}{\arabic{chapter}.\arabic{section}.\arabic{Lem}}
\newenvironment{Lem}[1][]{
	\refstepcounter{Lem}
	\mdfsetup{
		frametitle={
			\tikz[baseline=(current bounding box.east), outer sep=0pt]
			\node[anchor=east,rectangle,fill=gray!25]
			{\strut Lemma~\theLem\ifstrempty{#1}{}{:~#1}};},
		innertopmargin=10pt,linecolor=gray!25,
		linewidth=2pt,topline=true,
		frametitleaboveskip=\dimexpr-\ht\strutbox\relax
	}
	\begin{mdframed}[]\relax
}{\end{mdframed}}

%----------------------------------------------------------------------------------------
%	Corollary
%----------------------------------------------------------------------------------------
\newcounter{Cor}[section]
\renewcommand{\theCor}{\arabic{chapter}.\arabic{section}.\arabic{Cor}}
\newenvironment{Cor}[1][]{
	\refstepcounter{Cor}
	\mdfsetup{
		frametitle={
			\tikz[baseline=(current bounding box.east), outer sep=0pt]
			\node[anchor=east,rectangle,fill=gray!20]
			{\strut Corollary~\theCor\ifstrempty{#1}{}{:~#1}};},
		innertopmargin=10pt,linecolor=gray!20,
		linewidth=2pt,topline=true,
		frametitleaboveskip=\dimexpr-\ht\strutbox\relax
	}
	\begin{mdframed}[]\relax
}{\end{mdframed}}

%----------------------------------------------------------------------------------------
%	Example
%----------------------------------------------------------------------------------------
% \newtcbtheorem[no counter]{Example}{Example}{
%   enhanced,
%   sharp corners,
%   attach boxed title to top left={
%     yshifttext=-1mm
%   },
%   colback=white,
%   colframe=mygray,
%   fonttitle=\bfseries,
%   coltitle=white,
%   boxed title style={
%     sharp corners,
%     size=small,
%     colback=mygray,
%     colframe=mygray,
%   } 
% }{exa}

\newtcbtheorem[number within=section]{Example}{Example}
{
	colback = myexamplebg,
	breakable,
	colframe = myexamplefr,
	coltitle = myexampleti,
	boxrule = 1pt,
	sharp corners,
	detach title,
	before upper=\tcbtitle\par\smallskip,
	fonttitle = \bfseries,
	description font = \mdseries,
	separator sign none,
	description delimiters parenthesis,
}
{ex}

%----------------------------------------------------------------------------------------
%	Proposition
%----------------------------------------------------------------------------------------
\newtcbtheorem[number within=section]{Proposition}{Proposition}{
  enhanced,
  sharp corners,
  attach boxed title to top left={
    yshifttext=-1mm
  },
  colback=white,
  colframe=brownb,
  fonttitle = \bfseries\sffamily,
  coltitle=white,
  boxed title style={
    sharp corners,
    size=small,
    colback=brownb,
    colframe=brownb,
  } 
}{Prop}

%----------------------------------------------------------------------------------------
%	Hint
%----------------------------------------------------------------------------------------

%----------------------------------------------------------------------------------------
%	Property
%----------------------------------------------------------------------------------------
% \tcbuselibrary{theorems, skins, hooks}
% \newtcbtheorem[number within=section]{Property}{Property}
% {%
% 	enhanced,
% 	breakable,
% 	colback = mytheorembg,
% 	frame hidden,
% 	boxrule = 0sp,
% 	borderline west = {2pt}{0pt}{mytheoremfr},
% 	sharp corners,
% 	detach title,
% 	before upper = \tcbtitle\par\smallskip,
% 	coltitle=black,
% 	fonttitle = \bfseries,
% 	description font = \mdseries,
% 	separator sign none,
% 	segmentation style={solid, mytheoremfr},
% }
% {th}

% \tcbuselibrary{theorems, skins, hooks}
% \newtcbtheorem[number within=chapter]{property}{Property}
% {%
% 	enhanced,
% 	breakable,
% 	colback = mytheorembg,
% 	frame hidden,
% 	boxrule = 0sp,
% 	borderline west = {2pt}{0pt}{mytheoremfr},
% 	sharp corners,
% 	detach title,
% 	before upper = \tcbtitle\par\smallskip,
% 	coltitle = mytheoremfr,
% 	fonttitle = \bfseries\sffamily,
% 	description font = \mdseries,
% 	separator sign none,
% 	segmentation style={solid, mytheoremfr},
% }
% {th}


\tcbuselibrary{theorems,skins,hooks}
\newtcolorbox{Theoremcon}
{%
	enhanced
	,breakable
	,colback = mytheorembg
	,frame hidden
	,boxrule = 0sp
	,borderline west = {2pt}{0pt}{mytheoremfr}
	,sharp corners
	,description font = \mdseries
	,separator sign none
}

%----------------------------------------------------------------------------------------
%	claim
%----------------------------------------------------------------------------------------
% \theoremstyle{remark}
% \newtheorem{claim}{\bfseries Claim}
%\theoremstyle{remark}
%\newtheorem{claim}{\bfseries Claim}
%\newtheorem{newclaim}{\bfseries Claim}[section] % 新的定理环境

%----------------------------------------------------------------------------------------
%	Axiom
%----------------------------------------------------------------------------------------
\newtcbtheorem[no counter]{Axiom}{Axiom}{
  enhanced,
  sharp corners,
  attach boxed title to top left={
    yshifttext=-1mm
  },
  colback=white,
  colframe=bluea,
  fonttitle=\bfseries,
  coltitle=white,
  boxed title style={
    sharp corners,
    size=small,
    colback=bluea,
    colframe=bluea,
  } 
}{Axi}
%----------------------------------------------------------------------------------------
%	Note
%----------------------------------------------------------------------------------------
\usepackage{tikz}
\usepackage[most]{tcolorbox}

\usetikzlibrary{arrows,calc,shadows.blur}
\tcbuselibrary{skins}

\newtcolorbox{note}[1][]{%
    enhanced jigsaw,
    colback=gray!20!white,%
    colframe=gray!80!black,
    size=small,
    boxrule=1pt,
    title=\textbf{Note:-},
    halign title=flush center,
    coltitle=black,
    breakable,
    drop shadow=black!50!white,
    attach boxed title to top left={xshift=1cm,yshift=-\tcboxedtitleheight/2,yshifttext=-\tcboxedtitleheight/2},
    minipage boxed title=1.5cm,
    boxed title style={%
        colback=white,
        size=fbox,
        boxrule=1pt,
        boxsep=2pt,
        underlay={%
            \coordinate (dotA) at ($(interior.west) + (-0.5pt,0)$);
            \coordinate (dotB) at ($(interior.east) + (0.5pt,0)$);
            \begin{scope}
                \clip (interior.north west) rectangle ([xshift=3ex]interior.east);
                \filldraw [white, blur shadow={shadow opacity=60, shadow yshift=-.75ex}, rounded corners=2pt] (interior.north west) rectangle (interior.south east);
            \end{scope}
            \begin{scope}[gray!80!black]
                \fill (dotA) circle (2pt);
                \fill (dotB) circle (2pt);
            \end{scope}
        },
    },
    #1,
}
% ---------------------------------------------------------------------------------------
%    Exercise
%---------------------------------------------------------------------------------------
\newtcbtheorem[number within=section]{exercise}{Exercise}{enhanced, breakable,
colback = white, coltitle=black, colframe = black!25!white, colbacktitle = black!15!white,
attach boxed title to top left = {yshift = -2mm, xshift = 5mm},
boxed title style = {sharp corners},
fonttitle = \bfseries, separator sign = {.}, label type=exercise}{ex}
% \crefname{exercise}{exercise}{exercises}
% ---------------------------------------------------------------------------------------
%    Proof
%---------------------------------------------------------------------------------------
\usepackage{amsthm}
\renewcommand{\proofname}{\normalfont \textbf{Proof}}
%\renewenvironment{proof}{
%  \begin{leftbar}
%    \emph{\normalfont \textbf{Proof}.}
%}
%{
%  \newline
%  \null\hfill$\qed$
%  \end{leftbar}
%}
% \renewcommand\qedsymbol{Q.E.D.}
% \renewcommand\qedsymbol{$\blacksquare$}
\newenvironment{Proof of claim}
  {\begin{proof}[\normalfont \textbf{Proof of claim}]}
  {\end{proof}}
\newcommand\red[1]{\textcolor{red}{#1}}
\newenvironment{Soln}{\vspace{4pt}\textbf{\\}{\color{red}\textit{Solution}}.\normalfont}{\color{red}\hfill$\blacksquare$\vspace{4pt}\par}	% https://www.allacronyms.com/solution/abbreviated
\newenvironment{skPf}{\vspace{1ex}\textbf{\\}{\color{black}\textbf{Sketch of Proof}}.\normalfont}{\color{black}\hfill$\blacksquare$\vspace{1ex}\par}

\renewenvironment{pf of claim}{
  \begin{leftbar}
    \emph{\normalfont \textbf{Proof of Claim.}}
}
{
  \newline
  \null\hfill$\lrcorner$
  \end{leftbar}
}
%----------------------------------------------------------------------------------------
%	Basic Thm Environment
%----------------------------------------------------------------------------------------
\theoremstyle{definition}
\newtheorem*{Def}{Definition}
\newtheorem*{Prop}{Proposition}
\newtheorem*{Lemma}{Lemma}
\newtheorem*{Theo}{Theorem}
\newtheorem*{Coro}{Corollary}
\newtheorem*{remark}{Remark}
\newtheorem*{claim}{Claim}
\theoremstyle{plain}
\newtheorem{Ex}{Exercise}
\newtheorem{Prob}{Problem}

% =============== Filter unwanted warnings========================
\usepackage{silence}
\WarningsOff[tcolorbox]
\hbadness=1000000

\usepackage{fancyhdr}
\pagestyle{fancy}
\lhead{\vspace{-2ex}\includegraphics[height=3.956ex]{pictures/NJUPT_Math.png}}\setlength{\headheight}{3.8000ex}
% \lhead{\vspace{-2ex}\includegraphics[height=3.956ex]{images/NJUPT_Math.png}}\setlength{\headheight}{3.8000ex}
\chead{}
\rhead{\textsl{Notes} for \textbf{Abstract Algebra}}
\lfoot{\color{gray}\small \copyright\ \textsc{H.-T.~Guo}~(\texttt{guohuitongguo@gmail.com})}
\cfoot{\thepage}
\rfoot{\color{gray}\small 2023}
% \usepackage{fancyhdr}
% \usepackage{graphicx}
% \usepackage{xcolor}

% \pagestyle{fancy}
% \fancyhf{} % 清除现有的页眉和页脚

% % 使图片紧贴页眉线
% \lhead{\vspace{-1.ex}\raisebox{-1ex}{\includegraphics[height=3.956ex]{NJUPT_Math.png}}}\setlength{\headheight}{3.8000ex}

% \chead{}
% \rhead{Abstract Algebra}
% \lfoot{\color{gray}\small \copyright\ \textsc{T.-Y.~Li}~(\texttt{kellty@pku.edu.cn})}
% \cfoot{\thepage}
% \rfoot{\color{gray}\small 2021}

% % 调整页眉高度
% \setlength{\headheight}{4.2ex}
